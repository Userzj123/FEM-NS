%%%%%%%%%%%%%%%%%%%%%%%%%%%%%%%%%%%%%%%%%
% Diaz Essay
% LaTeX Template
% Version 2.0 (13/1/19)
%
% This template originates from:
% http://www.LaTeXTemplates.com
%
% Authors:
% Vel (vel@LaTeXTemplates.com)
% Nicolas Diaz (nsdiaz@uc.cl)
%
% License:
% CC BY-NC-SA 3.0 (http://creativecommons.org/licenses/by-nc-sa/3.0/)
%
%%%%%%%%%%%%%%%%%%%%%%%%%%%%%%%%%%%%%%%%%

%----------------------------------------------------------------------------------------
%	PACKAGES AND OTHER DOCUMENT CONFIGURATIONS
%----------------------------------------------------------------------------------------

\documentclass[11pt]{diazessay} % Font size (can be 10pt, 11pt or 12pt)
\usepackage{amsmath,amssymb}
\DeclareMathOperator{\E}{\mathbb{E}}
\usepackage{graphicx}
%----------------------------------------------------------------------------------------
%	TITLE SECTION
%----------------------------------------------------------------------------------------

\title{\textbf{Final Project of Numerical Methods for PDE} \\ {\Large\itshape Finite Volume Method for Euler Equation}} % Title and subtitle

\author{\textbf{Zejian You} \\ \textit{Columbia University}} % Author and institution

\date{\today} % Date, use \date{} for no date

%----------------------------------------------------------------------------------------

\begin{document}

\maketitle % Print the title section

%----------------------------------------------------------------------------------------
%	ABSTRACT AND KEYWORDS
%----------------------------------------------------------------------------------------

%\renewcommand{\abstractname}{Summary} % Uncomment to change the name of the abstract to something else

\begin{abstract}

\end{abstract}

\hspace*{3.6mm}\textit{Keywords:}  % Keywords

\vspace{30pt} % Vertical whitespace between the abstract and first section

%----------------------------------------------------------------------------------------
%	ESSAY BODY
%----------------------------------------------------------------------------------------
\section{Problem Description}

- Mass Conservation
  
$$\rho_t + (\rho u)_x = 0.$$

- Momentum Conservation

$$(\rho u)_t + (\rho u^2 + p)_x = 0.$$

- Energy Conservation
  
$$E = \rho e + \frac{1}{2}\rho u^2.$$
$$E_t + (u(E+p))_x = 0.$$

\subsection{Riemann Problem}
\subsubsection{Exact Riemann Problem}
\subsubsection{Approximate Riemann Problem}

\subsection{Conservative Flux}
\subsection{Lax-Friedrichs Numerical Flux}
\subsection{Refined Lax-Friedrichs Numerical Flux}
\subsubsection{Roe Numerical Flux}



\section{Methods}
\subsection{Finite Volume Method}

The governing equation could be written into matrix form:
$$
    \frac{\partial \bf{V}}{\partial t} + \frac{\partial \bf{F}(\bf{V})}{\partial x} =0\\
$$
$$
    \bf{V} = \begin{bmatrix}
        \rho \\ \rho u\\ \rho E
    \end{bmatrix}, \quad \bf{F}(\bf{V})=\begin{bmatrix}
        \rho u \\\frac{1}{2}\rho u^2 + p b\\(E + p) u
    \end{bmatrix}
$$

\section{Error Analysis}
\section{Conclusion}


%----------------------------------------------------------------------------------------
%	BIBLIOGRAPHY
%----------------------------------------------------------------------------------------
\bibliographystyle{unsrt}
\bibliography{../Library.bib}

%----------------------------------------------------------------------------------------

\end{document}
