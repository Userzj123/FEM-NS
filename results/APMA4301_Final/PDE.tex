%%%%%%%%%%%%%%%%%%%%%%%%%%%%%%%%%%%%%%%%%
% Diaz Essay
% LaTeX Template
% Version 2.0 (13/1/19)
%
% This template originates from:
% http://www.LaTeXTemplates.com
%
% Authors:
% Vel (vel@LaTeXTemplates.com)
% Nicolas Diaz (nsdiaz@uc.cl)
%
% License:
% CC BY-NC-SA 3.0 (http://creativecommons.org/licenses/by-nc-sa/3.0/)
%
%%%%%%%%%%%%%%%%%%%%%%%%%%%%%%%%%%%%%%%%%

%----------------------------------------------------------------------------------------
%	PACKAGES AND OTHER DOCUMENT CONFIGURATIONS
%----------------------------------------------------------------------------------------

\documentclass[11pt]{diazessay} % Font size (can be 10pt, 11pt or 12pt)
\usepackage{amsmath,amssymb}
\DeclareMathOperator{\E}{\mathbb{E}}
\usepackage{graphicx}
%----------------------------------------------------------------------------------------
%	TITLE SECTION
%----------------------------------------------------------------------------------------

\title{\textbf{Final Project of Numerical Methods for PDE} \\ {\Large\itshape Finite Volume Method for Euler Equation}} % Title and subtitle

\author{\textbf{Zejian You} \\ \textit{Columbia University}} % Author and institution

\date{\today} % Date, use \date{} for no date

%----------------------------------------------------------------------------------------

\begin{document}

\maketitle % Print the title section

%----------------------------------------------------------------------------------------
%	ABSTRACT AND KEYWORDS
%----------------------------------------------------------------------------------------

%\renewcommand{\abstractname}{Summary} % Uncomment to change the name of the abstract to something else
%----------------------------------------------------------------------------------------
%	ESSAY BODY
%----------------------------------------------------------------------------------------
\section{Problem Description}

- Mass Conservation
  
$$\rho_t + (\rho u)_x = 0.$$

- Momentum Conservation

$$(\rho u)_t + (\rho u^2 + p)_x = 0.$$

- Energy Conservation
  
$$E = \rho e + \frac{1}{2}\rho u^2.$$
$$E_t + (u(E+p))_x = 0.$$

To close the equation system, we need one more equation

- Equation of State
$$
p = \rho e(\gamma -1)
$$

\subsection{Hyperbolic structure of the 1D Euler equations}
The governing equation could be written into a system of hyperbolic equations:
$$
    \frac{\partial \bf{V}}{\partial t} + \frac{\partial \bf{F}(\bf{V})}{\partial x} =0 \quad \text{or}
$$
$$
    \bf{V} = \begin{bmatrix}
        \rho \\ \rho u\\ E
    \end{bmatrix}, \quad \bf{F}(\bf{V})=\begin{bmatrix}
        \rho u \\\rho u^2 + p\\(E + p) u
    \end{bmatrix}
$$


Introduce \textit{primitive variable} $\textbf{q} = [\rho, u, p]^{T}$ and write above equation in quasi-linear form with primitive variable:

$$
\begin{bmatrix}
    \rho \\ u\\ p
\end{bmatrix}_t
+
\begin{bmatrix}
    u & \rho & 0\\
    0 & u & 1/\rho\\
    0 & \gamma p & u
\end{bmatrix}
\begin{bmatrix}
    \rho\\ u\\ p
\end{bmatrix}_x
=0
$$

$$
\textbf{q}_t + \frac{\partial \bar{\textbf{F}}(\textbf{q})}{\partial \textbf{q}}\textbf{q}_x= 0
$$


\subsection{Finite Volume Method}
Based on Green's formula, we have

$$
\nabla\cdot \textbf{F}(\textbf{V}) = \oint_{\partial T_j} \textbf{F}(\textbf{V})\cdot \textbf{n} ds
$$

For finite volume discretization, we have\cite{li_multigrid_nodate}

$$
\oint_{\partial T_j} \textbf{F}(\textbf{V})\cdot \textbf{n} ds 
\approx \sum_{e_{ik}\in \partial T_j} \oint_{e_{jk}} \bar{\textbf{F}}(\textbf{V}_j, \textbf{V}_k)\cdot \textbf{n}_{jk} dl
$$


\begin{align}
    Q_j^{n+1} = Q_j^n - \frac{\Delta t_n}{\Delta x_j} (F_{j+1/2}^n - F_{j-1/2}^n)
\end{align}

\subsection{Eigenvalue and Eigenvector of Euler Flux Jacobian}

\begin{enumerate}
    \item Flux Jacobian of primitive variable is equal to 

    $$
    \begin{aligned}
        \frac{\partial \bar{\textbf{F}}(\textbf{q})}{\partial \textbf{q}}= \begin{bmatrix}
            u & \rho & 0\\
            0 & u & 1/\rho\\
            0 & \gamma p & u
        \end{bmatrix}
    \end{aligned}
    $$

    The eigenvalues and corresponding eigenvectors are

    $$
    \begin{aligned}
        & \lambda_1 = u -c, &&\lambda_2 = u,\quad && \lambda_3 = u+c\\
        &\textbf{r}_1 =\begin{bmatrix} -\rho /c\\1\\-\rho c\end{bmatrix}
        &&\textbf{r}_2 =\begin{bmatrix} 1 \\ 0 \\ 0 \end{bmatrix}\quad
        &&\textbf{r}_3 =\begin{bmatrix} \rho /c\\1\\\rho c\end{bmatrix}
    \end{aligned}
    $$

    \item Flux of variable $\textbf{V}$ equal to
    
    $$
    \begin{aligned}
        \bar{\textbf{F}}(\textbf{V})= \begin{bmatrix}
            V_2\\ 
            \frac{V_2^2}{V_1} + \left(V_3 - \frac{V_2^2}{2V_1}\right) (\gamma -1)\\
            \frac{\gamma V_2V_3}{V_1} - \frac{V_2^3}{2V_1^2}(\gamma -1)
        \end{bmatrix}
    \end{aligned}
    $$


    Flux Jacobian of variable $\textbf{V}$ equal to \footnote{with $p = (E - \frac{1}{2}\rho u^2)(\gamma -1)$, and $H=\frac{E+p}{\rho}$}

    $$
    \begin{aligned}
        \frac{\partial \bar{\textbf{F}}(\textbf{V})}{\partial \textbf{V}}= \begin{bmatrix}
            u & 1 & 0 \\
            \frac{\gamma -3}{2} u^2 & (3-\gamma)u & \gamma -1\\
            \frac{\gamma -1}{2}u^3-uH & H-(\gamma -1)u^2 & \gamma u\\
        \end{bmatrix}
    \end{aligned}
    $$

    Then the Euler Equation becomes 

    $$
    \textbf{V}_t + \frac{\partial \bar{\textbf{F}}(\textbf{V})}{\partial \textbf{V}}\textbf{V}_x= 0
    $$

    The eigenvalues and corresponding eigenvectors of Jacobian matrix is
    $$
    \begin{aligned}
        & \lambda_1 = u -c, &&\lambda_2 = u,\quad && \lambda_3 = u+c\\
        &\textbf{r}_1 =\begin{bmatrix} 1 \\ u-c \\H-uc\end{bmatrix}
        &&\textbf{r}_2 =\begin{bmatrix} 1 \\ u \\ \frac{1}{2}u^2 \end{bmatrix}\quad
        &&\textbf{r}_3 =\begin{bmatrix} 1\\ u+c \\H+uc\end{bmatrix}
    \end{aligned}
    $$

    Here $c=\sqrt{(\gamma - 1)(H-\frac{1}{2}u^2)}$.    \cite{david_i_ketcheson_chapter_2020}

\end{enumerate}
\begin{itemize}
    \item \textbf{linear degenerate}
\end{itemize}



\subsection{Conservative Flux}
\subsubsection{Lax-Friedrichs Numerical Flux}
\subsubsection{Refined Lax-Friedrichs Numerical Flux}
\subsubsection{Roe Numerical Flux}


\cite{roe_approximate_1981}

\begin{itemize}
    \item \textbf{Roe Average}
    \begin{equation}
        \bar{u} = \frac{\sqrt{\rho_r}u_r + \sqrt{\rho_l}u_l}{\sqrt{\rho_r}+\sqrt{\rho_l}}, \quad \bar{H} =\frac{\sqrt{\rho_r}H_r + \sqrt{\rho_l}H_l}{\sqrt{\rho_r}+\sqrt{\rho_l}}
    \end{equation}
\end{itemize}




\section{Riemann Problem}
\subsection{Exact Riemann Problem}

\begin{itemize}
    \item \textbf{Rankine-Hugoniot jump condition}
    
    For a shock wave connecting a known left or right state $q^*$ and an unknown middle state $q$, the moving speed $s$ of the shock should obey Eq.(\ref{eq3:RH jump condition}).\cite{david_i_ketcheson_chapter_2020}

\begin{equation}\label{eq3:RH jump condition}
    s(q^*-q) = f(q^*) - f(q)
\end{equation}

\item \textbf{Hugoniot locus}

\item \textbf{Lax entropy condition}

This physical entropy condition is equivalent to the mathematical condition that for a 1-shock to be physically relevant, the 1-characteristics must impinge on the shock (the Lax entropy condition). If the entropy condition is violated, the 1-characteristics would spread out, allowing the insertion of an expansion fan (rarefaction wave).
\begin{itemize}
    \item Shocks appear in regions where characteristics converge, as in the traffic jam example above.
    \item Rarefactions appear in regions where characteristics are spreading out, as in the green light example.\cite{david_i_ketcheson_chapter_2020}
\end{itemize}

\item \textbf{Integral curves}

\item \textbf{Riemann invariants}

\item \textbf{similarity solutions}
\item 
\end{itemize}



\cite{sanders_introduction_nodate}
\subsection{Approximate Riemann Problem}

Besides the exact Riemann solution, we could also obtain an approximate solution by linearizing the Euler Equation and replacing Jacobian of flux $\frac{\partial \bf{F}(\bf{V})}{\partial \bf{q}}$ by a linear operator $\hat{\bf{A}}(q_l, q_r)$ depending on the left and right status \cite{roe_approximate_1981}. Furthermore, this approximate linear operator should satisfy:


\begin{enumerate}

\item Consistency: $\hat{\bf{A}}(q_l, q_r) \rightarrow f'(q)$ as $q_l, q_r \rightarrow q$
\item Hyperbolicity: $\hat{\bf{A}}$ must be diagonalizable with real eigenvalues, so that we can define the waves and speeds needed in the approximate solution.
\item Conservation: $\hat{\bf{A}}(q_l, q_r)(q_r-q_l) = f(q_r) - f(q_l)$
\end{enumerate}


One common linear operator is the flux Jacobian of an average state:

\begin{equation}
    \hat{\bf{A}}(q_l, q_r) = f'(\hat{q})\text{, }\quad \hat{q}\in[q_l, q_r]
\end{equation}

For the system of hyperbolic equations, we could solve the system by decomposing the linear operator $\hat{\bf{A}}$ into the eigenvectors $\bf{R}$ and the matrix that eigenvalues on the diagonal $\bf{\Lambda}$ so that

$$
\begin{aligned}
    \bf{q}_t + \hat{\bf{A}} \bf{q}_x  &= 0\\
    \bf{q}_t + \bf{R}\Lambda\bf{R}^{-1}\bf{q}_x & = 0\\
    \bf{R}^{-1}\bf{q}_t + \Lambda\bf{R}^{-1}\bf{q}_x & = 0\\
    \bf{w}_t + \Lambda\bf{w}_x & = 0\\
    w_k(x, t) & = w_k(x-\lambda_k t, 0)\\
    \bf{q}(x, t) & = \sum_k w_k(x-\lambda_k t, 0) \bf{r_k}
\end{aligned}
$$


\subsection{Newton Iteration}
\subsubsection{Stationary Solution}

For the stationary solution, we would have following iteration scheme \cite{li_multigrid_nodate}

$$
\begin{aligned}
    \alpha \|\textbf{R}_j^{(n)}\|_{l^1} \delta \textbf{U}_j^{(n)} 
    + \sum_{e_jk \in \partial T_j} \int_{e_{jk}} \left(\frac{\partial \bar{\textbf{F}}^{(n)}}{\partial \textbf{U}_j} \delta \textbf{U}_j^{(n)}\right) \cdot \textbf{n}_{jk} dl 
    + \sum_{e_jk \in \partial T_j} \int_{e_{jk}} \left(\frac{\partial \bar{\textbf{F}}^{(n)}}{\partial \textbf{U}_k} \delta \textbf{U}_k^{(n)}\right) \cdot \textbf{n}_{jk} dl 
    = -\textbf{R}_j^{(n)}
\end{aligned}
$$

\begin{enumerate}
    \item Input $\textbf{U}^{(0)}$ as the initial guess, and set $n=0$
    \item Use an approximate solver for the system to get a $\delta \textbf{U}^{(n)}$
    \item update $\textbf{U}^{(n+1)}$ by $\textbf{U}^{(n)}+\tau\delta \textbf{U}^{(n)}$
    \item Reconstruct $\textbf{U}^{(n+1)}$ using its cell mean values to get a piecewise polynomial expression on each cell
    \item Check if the residual $\textbf{R}^{n+1}$ is small enough
\end{enumerate}

\subsubsection{PDE}

\subsection{Linear Multigrid Method for Jacobian Matrix}
\cite{jameson_solution_1983}

\section{Error Analysis}
\section{Conclusion}


%----------------------------------------------------------------------------------------
%	BIBLIOGRAPHY
%----------------------------------------------------------------------------------------
\bibliographystyle{unsrt}
\bibliography{../Euler.bib}

%----------------------------------------------------------------------------------------

\end{document}
